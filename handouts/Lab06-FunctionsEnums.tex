\documentclass[12pt]{scrartcl}


\setlength{\parindent}{0pt}
\setlength{\parskip}{.25cm}

\usepackage{graphicx}

\usepackage{xcolor}

\definecolor{darkred}{rgb}{0.5,0,0}
\definecolor{darkgreen}{rgb}{0,0.5,0}
\usepackage{hyperref}
\hypersetup{
  letterpaper,
  colorlinks,
  linkcolor=red,
  citecolor=darkgreen,
  menucolor=darkred,
  urlcolor=blue,
  pdfpagemode=none,
  pdftitle={Introduction To Git},
  pdfauthor={Christopher M. Bourke},
  pdfcreator={$ $Id: cv-us.tex,v 1.28 2009/01/01 00:00:00 cbourke Exp $ $},
  pdfsubject={PhD Thesis},
  pdfkeywords={}
}

\definecolor{MyDarkBlue}{rgb}{0,0.08,0.45}
\definecolor{MyDarkRed}{rgb}{0.45,0.08,0}
\definecolor{MyDarkGreen}{rgb}{0.08,0.45,0.08}

\definecolor{mintedBackground}{rgb}{0.95,0.95,0.95}
\definecolor{mintedInlineBackground}{rgb}{.90,.90,1}

%\usepackage{newfloat}
\usepackage[newfloat=true]{minted}
\setminted{mathescape,
               linenos,
               autogobble,
               frame=none,
               framesep=2mm,
               framerule=0.4pt,
               %label=foo,
               xleftmargin=2em,
               xrightmargin=0em,
               startinline=true,  %PHP only, allow it to omit the PHP Tags *** with this option, variables using dollar sign in comments are treated as latex math
               numbersep=10pt, %gap between line numbers and start of line
               style=default, %syntax highlighting style, default is "default"
               			    %gallery: http://help.farbox.com/pygments.html
			    	    %list available: pygmentize -L styles
               bgcolor=mintedBackground} %prevents breaking across pages
               
\setmintedinline{bgcolor={mintedBackground}}
\setminted[text]{bgcolor={mintedBackground},linenos=false,autogobble,xleftmargin=1em}
%\setminted[php]{bgcolor=mintedBackgroundPHP} %startinline=True}
\SetupFloatingEnvironment{listing}{name=Code Sample}
\SetupFloatingEnvironment{listing}{listname=List of Code Samples}

\title{CSCE 155 - C}
\subtitle{Lab 06 - Functions, Passing By Reference,\\ and Enumerated Types}
\author{Dr.\ Chris Bourke}
\date{~}

\begin{document}

\maketitle

\section*{Prior to Lab}

Before attending this lab:
\begin{enumerate}
  \item Read and familiarize yourself with this handout.
  \item Review the lecture notes on conditionals, or the following free textbook resources:
	\begin{itemize}
  	  \item Functions: \url{http://www.cs.cf.ac.uk/Dave/C/node8.html}
	  \item Enumerated Types: \url{http://www.cs.cf.ac.uk/Dave/C/node9.html}
	  \item Pointers: \url{http://www.cs.cf.ac.uk/Dave/C/node10.html}
	\end{itemize}
\end{enumerate}

\section*{Peer Programming Pair-Up}

To encourage collaboration and a team environment, labs will be
structured in a \emph{pair programming} setup.  At the start of
each lab, you will be randomly paired up with another student 
(conflicts such as absences will be dealt with by the lab instructor).
One of you will be designated the \emph{driver} and the other
the \emph{navigator}.  

The navigator will be responsible for reading the instructions and
telling the driver what to do next.  The driver will be in charge of the
keyboard and workstation.  Both driver and navigator are responsible
for suggesting fixes and solutions together.  Neither the navigator
nor the driver is ``in charge.''  Beyond your immediate pairing, you
are encouraged to help and interact and with other pairs in the lab.

Each week you should alternate: if you were a driver last week, 
be a navigator next, etc.  Resolve any issues (you were both drivers
last week) within your pair.  Ask the lab instructor to resolve issues
only when you cannot come to a consensus.  

Because of the peer programming setup of labs, it is absolutely 
essential that you complete any pre-lab activities and familiarize
yourself with the handouts prior to coming to lab.  Failure to do
so will negatively impact your ability to collaborate and work with 
others which may mean that you will not be able to complete the
lab.  

\section{Lab Objectives \& Topics}
At the end of this lab you should be familiar with the following
\begin{itemize}
  \item Basics of enumerated types
  \item Functions with pass-by-value and pass-by-reference parameters
  \item How to design and use functions with error handling
  \item How to modularize code into separate header and source files
  \item The proper use of enumerated types and functions in solving a 
  given problem 
\end{itemize}

\section{Background}

\subsection*{Enumerated Types}

Enumerated types are data types that define a set of named values.  
Enumerated types are often ordered and internally associated with 
integers (starting with 0 by default and incremented by one in the 
order of the list).  The purpose of enumerated types is to organize 
certain types and enforce specific values.  Without enumerated types, 
integer values must be used and the convention of assigning certain 
integer values to certain types in a logical collection must be remembered 
or documentation referred to as needed.  Enumerated types provide 
a human-readable ``tag'' to these types of elements, relieving the 
programmer of continually having to refer to the convention and avoiding 
errors.

\subsection*{Passing by Value versus Passing by Reference}

In general there are two ways in which arguments can be passed to 
a function: by value or by reference. When an argument (a variable) 
is passed by value to a function, a copy of the value stored in the 
variable is placed on the program stack and the function uses this 
copy's value during its execution.  Any changes to the copy are not 
realized in the calling function but are local to the function.

In contrast, when an argument (variable) is passed by reference to a 
function, the variable's memory address is used.  Locally, the variable 
is treated as a pointer; to get or set its value, it must be dereferenced 
using the \mintinline{c}{*} operator.  Any changes to the memory cell pointed to by 
that pointer are reflected globally in the program and more specifically, 
in the calling function.

A full example:

\begin{minted}{c}
//a regular variable:
int a = 10; 

//a pointer to an integer, initially pointing to NULL
int *ptr = NULL; 

// & is the referencing operator, it gets the memory 
// address of the variable a so that ptr can point to it
ptr = &a; 

// * is the dereferencing variable: it changes the pointer
// back into a regular variable so a value can be assigned
// or accessed:
*ptr = 20; //a now holds the value 20
\end{minted}

One of the main uses of pointers is to enable pass-by-reference 
for functions.  Consider the following function:

\begin{minted}{c}
int foo(int a, int b) {
  a = 10;
  return a + b;
}
\end{minted}

When we call \mintinline{c}{foo}, the variable's we pass to it are passed
\emph{by value} meaning that the value stored in the variables at the
time that we call the function are copied and given to the function.  The
change to the variable \mintinline{c}{a} in line 2 \emph{only affects
the local variable}.  The original variable is unaltered.  That is, the following
code:

\begin{minted}{c}
int x = 5;
int y = 20;
int z = foo(x, y);
printf("x, y, z = %d, %d, %d\n" x, y, z);
\end{minted}

would print \mintinline{text}{5, 20, 30}: the variable \mintinline{c}{x} is not
changed by the function \mintinline{c}{foo}.  Contrast this with the following:

\begin{minted}{c}
int bar(int *a, int b) {
  *a = 10;
  return (*a) + b;
}
\end{minted}

In the function \mintinline{c}{bar}, the variable \mintinline{c}{a} is passed
\emph{by reference} using a pointer.  Since the memory address is passed
to \mintinline{c}{bar} instead of a copy of a value, we can alter the original
value.  Now when line 2 changes the value of the pointer \mintinline{c}{a}, 
the change affects the original variable.  That is, the code 

\begin{minted}{c}
int x = 5;
int y = 20;
int z = foo(&x, y);
printf("x, y, z = %d, %d, %d\n" x, y, z);
\end{minted}

Will now print \mintinline{text}{10, 20, 30} since the variable \mintinline{c}{x}
was passed by reference.

Passing by reference means that we can ``return'' multiple values and/or use
the return value to indicate an error code instead of a value now.

\section{Activities}

Clone the GitHub project for this lab using the following URL:
\url{https://github.com/cbourke/CSCE155-C-Lab06}

A flower shop sells various arrangements of a dozen flowers 
(roses, lilies, daisies) in two colors each (red or white) with a 
choice of bouquet or vase.  You are given an (incomplete) source 
file, \mintinline{text}{florist.c}, of a program that takes the order 
for a flower arrangement from the user and displays the cost. 
The program uses three enumerated types to define the type 
of flowers, color and flower arrangement.  Your task is to 
complete the program and implement the getCost function to 
compute the cost based on the following rules.
\begin{enumerate}
  \item The base price for each of the flowers is described in Table \ref{table:basePrice}
  \begin{table}[h]
  \centering
  \begin{tabular}{l|l}
  Flower & Price\\
  \hline\hline
  Roses & \$30 \\
  Lilies	 & \$20 \\
  Daisies & \$45 \\
  \end{tabular}
  \caption{Flower Prices}
  \label{table:basePrice}
  \end{table}
  \item Red Lilies and Red Daisies have an added cost of \$5 and White 
  	Roses have an added cost of \$10.  There is no additional cost for 
	other flower/color combinations
  \item Bouquets are free of charge while vases add an additional \$10 charge to the total
\end{enumerate}
	
\subsection{Defining Enumerated Types}

To complete the program, perform the following tasks.
\begin{enumerate}
  \item Insert the required items in the 3 enumerated types.  The \mintinline{c}{Color} 
  	enumerated type has already been completed for you.  Note: the rest of the 
	program assumes that all enumerated types begin with 1, you should keep 
	this assumption so that you don?t need to change the block of \mintinline{c}{printf}
	and \mintinline{c}{scanf} menu statements.
  \item After taking input from the user, the program calls the function 
	\mintinline{c}{getCost} to determine the cost. This function takes three 
	input parameters, the flower type, the color and the arrangement and 
	returns the cost (a \mintinline{c}{double}).
	\begin{enumerate}
	  \item The function prototype has been completed for you. Implement the 
	  definition of this function to return the cost of a given arrangement using 
	  the costs shown on the price list.
	  \item In the main function, declare the other required variable(s), complete 
	  the function call with appropriate arguments and print the cost. What's the 
	  data type of the variable cost?
	\end{enumerate}
  \item Compile using  \mintinline{text}{gcc -o florist florist.c} and test your program.  
	Answer the questions in your handout.
\end{enumerate}

\subsection{Passing By Reference}

Due to a harsh winter, Red Daisies are no longer available.  We will make the 
appropriate changes to the \mintinline{c}{getCost} function to accommodate this 
change and to communicate errors through the function's return type.  The cost 
will be communicated to the calling function by reference rather than as its return 
value.
\begin{enumerate}
  \item Change the signature of the getCost function to: \\
	\mintinline{c}{int getCost(double *cost, Flower flower, Color  color, Arrangement arr)}
  \item Change the definition of the function to do the following:
  	\begin{enumerate}
	  \item If given invalid arguments, it returns a value of 1 (indicating an error)
	  \item If given valid arguments, it returns a value of 0 (indicating no error) and 
		places the cost in the cost variable (which is passed by reference)
	\end{enumerate}
  \item In the \mintinline{c}{main} function, make other necessary changes to 
	accommodate the above.  In addition, you should check the return value 
	of the \mintinline{c}{getCost} function and print an error message in the case 
	that it results in an error. 
  \item Test your changes and demonstrate them to a lab instructor.
\end{enumerate}

\subsection{Using Error Codes}

Enumerated types are useful, but safety is not always guaranteed.  To illustrate, 
the following code would compile: \mintinline{c}{Color c = 10;} even though there is no color 
corresponding to the integer 10.  What happens if you call your \mintinline{c}{getCost} 
function using such a color?  To prevent this, we will add additional code to detect 
and handle this type of error situation.

\begin{enumerate}
  \item Add another enumerated type called \mintinline{c}{Error} that will 
	recognize the following error types: No Error, Unavailable Selection Error, 
	Invalid Argument Error, and Null Pointer Error (note: a good convention 
	is to use all-caps with words separated by an underscore).
  \item Add additional code and checks to your \mintinline{c}{getCost} function 
	to check for valid input (on all the arguments).
  \item Modify the \mintinline{c}{getCost} method to return one of your enumerated types 
	based on the type of error encountered.
  \item Modify your \mintinline{c}{main} function to check for the type of error returned and 
	print an appropriate error message
  \item Demonstrate your working program to a lab instructor
\end{enumerate}
	
\section{Modularizing Your Code}

Make your code more modular by organizing components into separate files.
\begin{enumerate}
  \item Place all of your function prototypes and enum definitions in a 
  	header file, \mintinline{text}{flowers.h}
  \item Place all of your function definitions into a source file, \mintinline{text}{flowers.c}
  \item Add the appropriate preprocessor directive to \mintinline{text}{flowers.c} and \mintinline{text}{florist.c}
  \item Compile and link your files:
  \begin{enumerate}
    \item Compile (but do not link) the \mintinline{text}{flowers.c} source file 
    	into an object file using: \mintinline{text}{gcc -c -o flowers.o flowers.c}
    \item Compile your \mintinline{text}{florist.c} source and link into the 
	\mintinline{text}{flowers.o} object file using: \mintinline{text}{gcc -o florist flowers.o florist.c}
  \end{enumerate}
  \item Demonstrate your results to a lab instructor. 
\end{enumerate}
  
\section{Advanced Activity 1 (Optional)}

In the C language, there is no built-in or standard mechanism to print 
a human-readable value of an enumerated type.  For example, given 
an enumeration:

\begin{minted}{c}
enum { 
  RED = 1, 
  WHITE 
} Color;
\end{minted} 

there is no built-in function to print the enumeration constant ``red'' 
or ``white'' given a \mintinline{c}{Color} type.

Modify your program by adding a function or functions so that you 
are able to print he labels of the enumerated items.  As a hint: the 
function signature for \mintinline{c}{Color} should be something like
\mintinline{c}{char * getName(Color c);}

\section{Advanced Activity 2 (Optional)}

Large projects require even more abstraction and tools to manage 
source code and specify how it gets built.  One tool to facilitate this 
is a program called Make which uses makefiles--specifications for 
which pieces of code depend on other pieces and how all pieces 
get built and combined to build an executable program (or collection 
of executables).  The use of modern IDEs has made the build process 
much more manageable, but make is still widely used.  Several 
useful tutorials exist describing how makefiles can be defined and 
used.
\begin{itemize}
  \item \url{http://www.opussoftware.com/tutorial/TutMakefile.htm}
  \item \url{http://mrbook.org/tutorials/make/}
\end{itemize}
Below is an example makefile.  Copy and paste this text into a file 
called \mintinline{text}{makefile} in the same directory as your other 
code.  From the command line, type \mintinline{text}{make} and 
observe that all of the elements of your program are built automatically!

\begin{minted}{make}
#Example makefile
#define a variable, CC which defines the compiler and its flags
#so that we can reuse it
CC = gcc -g -Wall

florist: florist.c flowers.o
  $(CC) -o florist flowers.o florist.c

flowers.o: flowers.h flowers.c
  $(CC) -c -o flowers.o flowers.c

clean:
  rm -f *.o *~ florist
\end{minted}

\end{document}
